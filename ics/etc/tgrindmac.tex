% @(#)tgrindmac.tex	1.4 (LBL) 3/30/85
% Macros for TeX "tgrind" (a TeX equivalent of 4bsd "vgrind").
%
%  Copyright (C) 1985 by Van Jacobson, Lawrence Berkeley Laboratory.
%  This program may be freely used and copied but may not be sold
%  without the author's written permission.  This notice must remain
%  in any copy or derivative.
%
%  Please send improvements, bug fixes, comments, etc., to 
%    van@lbl-rtsg.arpa
%    ...ucbvax!lbl-csam!van
%
%  Modifications.
%  --------------
%  10Feb85, vj	Written.
%  23Mar85, rf  Substitute ambx10 for amb10
%  29Mar85, Chris Torek: Use tt font for all characters in strings.
%		Print decent quotes in comments rather than use tt
%		font quotes.  Show filename (to terminal & log file)
%		at start of each new file.
%  30Mar85, vj	Fixed bug in tabbing.
%  7 September 1987, J. Greg Davidson
%	Butchered for including programs in documents.
%	All fonts specified in document macros.

% tfontedpr outputs a "\Head{Hdr text}" if you give it the "-h" flag.
% We remember the text in "\Header" so it can be included in the 
% head line.
\def\Head#1{\def\Header{#1}}
\def\Header{\null}

% We get a "\File{Filename},{Last Mod Time},{Last Mod Date}" at the start of
% each new file.  We remember this stuff for inclusion in the page head & foot.
% We reset the page number & current line number to 0 and output a null
% mark to let the output routine know we're starting a new file.
% We set up the \headline & \footline token lists inside the File macro to
% save remembering the filename & mod time with yet other macros.
\def\File#1,#2,#3{
	\gdef\FileName{#1}
	\def\FileName{#1}
	\mark{\empty}
	\global\linecount=0\linenext=9\
	\message{#1}
	% \vfill\eject
	% pageno=1 % jgd
	% \headline={\twelvebf\Header\hfil
	% \edef\a{\topmark}\edef\b{\botmark}\edef\c{\firstmark}
	% \ifx\c\empty\botmark\else
	% \ifx\a\empty\botmark\else
	% \ifx\b\empty\topmark\else
	% \ifx\a\b\topmark\else\topmark--\botmark\fi
	% \fi\fi\fi(#1)}
	% \footline={\it{}#2 #3\hfil{}Page \folio{} of #1}
}

% There's a "\Proc{Proc Name}" at the start of each procedure.  If
% the language definition allows nested procedures (e.g., pascal), there
% will be a "\ProcCont{Proc Name}" at the end of each inner procedure.
% (In this case, "proc name" is the name of the outer procedure.  I.e.,
% ProcCont marks the continuation of "proc name").
\def\Proc#1{\global\def\Procname{#1}\global\setbox\procbox=\hbox{\forteenrm #1}}
\def\ProcCont#1{\global\def\Procname{#1}
\global\setbox\procbox=\hbox{\forteenrm$\ldots$#1}}
\newbox\procbox
\def\Procname{\null}

% Each formfeed in the input is replaced by a "\NewPage" macro.  If
% you really want a page break here, define this as "\vfill\eject".
% \def\NewPage{\filbreak\bigskip}
\def\NewPage{\vfil\eject\def\TitleLine{\ContTitle}}

% Each line of the program text is enclosed by a "\L{...}".  We turn
% each line into an hbox of size hsize.  If we saw a procedure name somewhere
% in the line (i.e., "procbox" is not null), we right justify "procbox"
% on the line.  Every 10 lines we output a small, right justified line number.
\def\L#1{\filbreak\hbox to \hsize{\CF\strut\global\advance\linecount by1
#1\hss\linebox}} % old version was: -jgd
% #1\hss\ifvoid\procbox\linebox\else\box\procbox\mark{\Procname}\fi}}

\newcount\linecount \linecount=0
\newcount\linenext \linenext=9
\def\linebox{} % jgd
% \def\linebox{\ifnum\linecount>\linenext\global\advance\linenext by10
% \hbox{\sevenrm\the\linecount}\fi}


% The following weirdness is to deal with tabs.  "Pieces" of a line
% between tabs are output as "\LB{...}".  E.g., a line with a tab at
% column 16 would be output as "\LB{xxx}\Tab{16}\LB{yyy}".  (Actually, to
% reduce the number of characters in the .tex file the \Tab macro
% supplies the 2nd & subsequent \LB's.) We accumulate the LB stuff in an
% hbox.  When we see a Tab, we grab this hbox (using "\lastbox") and turn
% it into a box that extends to the tab position.  We stash this box in
% "\linesofar" & use "\everyhbox" to get \linesofar concatenated onto the
% front of the next piece of the line.  (There must be a better way of
% doing tabs [cf., the Plain.tex tab macros] but I'm not not enough of a
% TeX wizard to come up with it.  Suggestions would be appreciated.)

\def\LB{\CF\hbox}
\newbox\linesofar\setbox\linesofar=\null
\everyhbox={\box\linesofar}
\newdimen\TBwid
\def\Tab#1{\setbox\tbox=\lastbox\TBwid=1\wd\tbox\advance\TBwid by 1\ts
\ifdim\TBwid>#1\ts
\setbox\linesofar=\hbox{\box\tbox\space}\else
\setbox\linesofar=\hbox to #1\ts{\box\tbox\hfil}\fi\LB}

% A normal space is too thin for code listings.  We make spaces & tabs
% be in "\ts" units (which are the width of a "0" in the current font).
\newdimen\ts
\newbox\tbox
\setbox\tbox=\hbox{0} \ts=1\wd\tbox \setbox\tbox=\hbox{\hskip 1\ts}
\def\space{\hskip 1\ts\relax}

% Font changing stuff for keywords, comments & strings.  We put keywords
% in boldface, comments in text-italic & strings in typewriter.  Since
% we're usually changing the font inside of a \LB macro, we remember the
% current font in \CF & stick a \CF at the start of each new box.
% Also, the characters " and ' behave differently in comments than in
% code, and others behave differently in strings than in code.
\newif\ifcomment\newif\ifstring
\let\CF=\rm
\def\K#1{{\bf #1}}	% Keyword
\def\C{\it\global\let\CF=\it\global\commenttrue\relax}	% Comment Start
\def\CE{\rm\global\let\CF=\rm\global\commentfalse\relax}% Comment End
\def\S{\tt\global\let\CF=\tt\global\stringtrue\relax}	% String Start
\def\SE{\rm\global\let\CF=\rm\global\stringfalse\relax}	% String End

% Special characters.
\def\{{\ifmmode\lbrace\else\ifstring{\char'173}\else$\lbrace$\fi\fi}
\def\}{\ifmmode\rbrace\else\ifstring{\char'175}\else$\rbrace$\fi\fi}
\def\!{\ifmmode\backslash\else\ifstring{\char'134}\else$\backslash$\fi\fi}
\def\|{\ifmmode|\else\ifstring{\char'174}\else$|$\fi\fi}
\def\<{\ifmmode<\else\ifstring<\else$<$\fi\fi}
\def\>{\ifmmode>\else\ifstring>\else$>$\fi\fi}
\def\/{\ifmmode/\else\ifstring/\else$/$\fi\fi}
\def\-{\ifmmode-\else\ifstring-\else$-$\fi\fi}
\def\_{\ifstring{\char'137}\else\underbar{\ }\fi}
\def\&{{\char'046}}
\def\#{{\char'043}}
\def\%{{\char'045}}
\def\~{{\char'176}}
\def\"{\ifcomment''\else{\tt\char'042}\fi}
\def\'{\ifcomment'\else{\tt\char'047}\fi}
\def\^{{\char'136}}
\def\${{\rm\char'044}}

\raggedright\obeyspaces\let =\space%
